\chapter{التعامل مع القوائم في بايثون}
القوائم في لغة بايثون هي عبارة عن مجموعة من البيانات التي يشار اليها بمتغير واحد. وهي تعتبر وسيلة سهلة لتخزين البيانات قبل معالجتها وتحليلها. واقرب مثال يوضح اهمية القوائم هو ان يكون هناك عدة قيم لمتغير ما كدرجات الحرارة خلال السنة مثلا. فالطريقة  التي تعلمناها في الفصل السابق تجبرنا على كتابة متغير لكل قيمة كالاتي:

\begin{english}
\pythoncode{code/list1.py}
\end{english}

 هذه الطريقة طبعا متعبة ومملة. لذلك فان الطريقة الاسهل تتمثل في تخصيص قائمة بمتغير واحد حيث يتم الاشارة الى كل قيمة في هذه القائمة برقم يدل على موقعها. ومن الامثلة على هذه الطريقة مايلي:
\begin{english}
\pythoncode{code/list2.py}
\end{english}
\section{تاريخ}